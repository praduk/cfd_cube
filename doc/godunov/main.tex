\documentclass{doc}

\usepackage{tikz}
\usetikzlibrary{cd, matrix, calc, arrows}

\newcommand{\forceindent}{\leavevmode{\parindent=1em\indent}}

\title{2D Finite Volume Discretization Notes}
\author{P. Kannan}
\classification{}
\logoraw{CFDBox}


\begin{document}

\section{Conservation of Mass}
Let $\rho$ be the flow property density and $v$ the flow property velocity.
\begin{align}
    0 &= \pd{\rho}{t} + (\rho \tp{v}) \ldel_r \\
    \pd{}{t}\iiint_\Omega \rho\ d\Omega &= -\oiint_\Gamma \rho \tp{v} n\ d\Gamma
\end{align}

\section{Conservation of Momentum}
\begin{align}
    \sigma \ldel_r &= \pd{\rho v}{t} + (\rho v\tp{v}) \ldel_r \\
    \pd{}{t} \iiint_\Omega \rho v\ d\Omega &= \oiint_\Gamma \left( \sigma- \rho v\tp{v} \right)n\ d\Gamma
\end{align}
Note that the Cauchy stress tensor is
\begin{equation}
\sigma = -p I_3 + \tau
\end{equation}
where $p$ is pressure and $\tau$ is the deviatoric stress tensor defined by
\begin{equation}
\tau = \mu \begin{bmatrix}
    \pd{v^x}{x} - \pd{v^y}{y} & \pd{v^x}{y} + \pd{v^y}{x} \\
    \pd{v^x}{y} + \pd{v^y}{x} & \pd{v^y}{y} - \pd{v^x}{x} \\
\end{bmatrix}.
\end{equation}
Also note that
\begin{align}
    \sigma \ldel_r &= (-p I_3 + \tau ) \\
        &= \tp{-(\rdel_r p)} + \tau \ldel_r.
\end{align}

\section{Conservation of Total Energy (Ideal Gas)}
Define \emph{total energy} $E=\rho \left( u + \frac{1}{2}\tp{v}{v} \right)$
where $u$ is the internal energy.  Total energy is a conserved flow property.
\begin{align}
    s_E &= \pd{E}{t} + (E \tp{v}) \ldel_r \\
    \pd{}{t}\iiint_\Omega E\ d\Omega &= - \oiint_\Gamma (E \tp{v}) n\ d\Gamma + \iiint_\Omega s_u\ d\Omega  \\
    \pd{}{t}\iiint_\Omega E\ d\Omega &= \oiint_\Gamma \left( -(E \tp{v})
        - k_T \tp{(\rdel_r T)} + \tp{v}\sigma \right)n\ d\Omega
\end{align}
Note for an ideal gas at constant volume,
\begin{align}
    u &= c_v T = \frac{E}{\rho} - \frac{1}{2} \tp{v}v \\
    p &= \rho R T = \rho (c_p-c_v) \frac{u}{c_v} = \rho(\gamma-1) u.
\end{align}

\section{Cleaning up Equations of Flow}
Let
\begin{equation}
    \pd{}{t} \iiint_\Omega \phi\ d\Omega + \oiint_\Gamma f(\phi) n \ d\Gamma = 0.
\end{equation}
Define $\phi$ as
\begin{equation}
    \phi = \begin{bmatrix}
        \rho \\
        \rho v \\
        E
    \end{bmatrix}.
\end{equation}
and $f$ as
\begin{equation}
    f = \begin{bmatrix}
        \rho \tp{v} \\
        -\sigma + \rho v \tp{v} \\
        E\tp{v} + k_T \tp{(\rdel_r T)} - \tp{v} \sigma
    \end{bmatrix}.
\end{equation}
Let $f^x$ and $f^y$ denote the first and second columns of $f$:
\begin{align}
    f^x = \begin{bmatrix}
        \rho v^x \\
        \begin{bmatrix}
            -\sigma^{xx} + \rho v^{x}v^{x} \\
            -\sigma^{xy} + \rho v^{y}v^{x}
        \end{bmatrix} \\
        Ev^x + k_T \pd{T}{x} - \left( \sigma^{xx} v^x - \sigma^{xy} v^{y} \right)
    \end{bmatrix}. \\
    f^y = \begin{bmatrix}
        \rho v^y \\
        \begin{bmatrix}
            -\sigma^{xy} + \rho v^{x}v^{y} \\
            -\sigma^{yy} + \rho v^{y}v^{y}
        \end{bmatrix} \\
        Ev^y + k_T \pd{T}{y} - \left( \sigma^{yy} v^y - \sigma^{xy} v^{x} \right)
    \end{bmatrix}.
\end{align}

\section{2D First-Order Finite Volume Discretization}
For a 2D-square, let $\Delta w$ denote the width of the square.
Let $F_{i+1/2,j}, F_{i-1/2,j}$ denote the flux of $f$ on
the left and right faces of cube $(i,j)$.
Let $F_{i,j+1/2}, F_{i,j-1/2}$ denote the flux of $f$ on
the lower and upper faces of cube $(i,j)$.
%In particular
%\begin{align}
%
%\end{align}
Define the cell averaged value of flow property $\phi$ as
\begin{equation}
    \bar{\phi} = \frac{\iiint_\Omega \phi\ d\Omega}{\iiint_\Omega \ d\Omega}.
\end{equation}
Then,
\begin{align}
    0 &= (\Delta w)^2 \pd{\bar{\phi}}{t} + \oiint_\Gamma f(\phi) n \ d\Gamma \\
    0 &= (\Delta w)^2 \pd{\bar{\phi}}{t} + \Delta w \left(
        F_{i+1/2,j} \begin{bmatrix} 1\\  0 \end{bmatrix} +
        F_{i-1/2,j} \begin{bmatrix}-1\\  0 \end{bmatrix} +
        F_{i,j+1/2} \begin{bmatrix} 0\\  1 \end{bmatrix} +
        F_{i,j-1/2} \begin{bmatrix} 0\\ -1 \end{bmatrix}
    \right) \\
    0 &= \pd{\bar{\phi}}{t} + \frac{1}{\Delta w} \left(
        F^x_{i+1/2,j} -
        F^x_{i-1/2,j} +
        F^y_{i,j+1/2} -
        F^y_{i,j-1/2} 
    \right).
\end{align}
If performing Euler integration,
\begin{equation}
    \Delta \bar{\phi} = -\frac{\Delta t}{\Delta w} \left(
        F^x_{i+1/2,j} -
        F^x_{i-1/2,j} +
        F^y_{i,j+1/2} -
        F^y_{i,j-1/2} 
    \right).
\end{equation}
Or, flipping signs
\begin{equation}
    \Delta \bar{\phi} = \frac{\Delta t}{\Delta w} \left(
        F^x_{i-1/2,j} -
        F^x_{i+1/2,j} +
        F^y_{i,j-1/2} -
        F^y_{i,j+1/2}
    \right).
\end{equation}

\section{Classical First-Order Upwind Godunov Scheme
         with Roe-Approximate Riemann Solver}
Russian mathematician S.K. Godunov developed the Godunov scheme in 1959 and it is
a stable scheme that can properly handle shocks and discontiunities.

\forceindent In Godunov's scheme, each control volume is approximated to have uniform
flow properties with discontinuties at cell interfaces.  The flux through each
cell interface is analyzed individually as a ``Riemann problem''.  For example for an x-oriented
interface
\begin{align}
    \pd{\phi}{t} + \pd{f^x(\phi)}{x} = 0.
\end{align}
Define $\phi_R$ to be the initial uniform flow property on the right side of the interface and
$\phi_L$ the intial uniform flow property on the left side of the interface.
Define $\tilde{\phi}$ as
\begin{equation}
    \tilde{\phi} = w \phi_L + (1-w) \phi_R\text{ for some $w\in [0, 1]$}.
\end{equation}
The Roe-approximate Riemannn solver chooses
\begin{equation}
    w = \frac{\sqrt{\rho_L}}{\sqrt{\rho_L} + \sqrt{\rho_R}},
\end{equation}
where $\rho$ is density.  The Roe-approximate Riemann solver then linearizes the gas dynamics about
the interface as
\begin{align}
    \pd{\phi}{t} + \underbrace{\pd{f^x}{\phi}}_\text{\hidewidth Assumed Constant \hidewidth} \pd{\phi}{x} = 0
\end{align}
so that
\begin{equation}
    \pd{\phi}{t} + A \pd{\phi}{x} = 0
\end{equation}
where
\begin{align}
    A &= \pd{f^x(\tilde\phi)}{\phi}.
\end{align}
Let the eigen-decomposition of be $A=S\Lambda S^{-1}$ and 
$\lambda_i$ be each diagonal entry of $\Lambda$.  Let
the eigen-states be
\begin{align}
    \psi &= S^{-1} \phi.
\end{align}
Define $\tilde{\psi}$ by
\begin{equation}
    \tilde{\psi}_i = \begin{cases}
        \psi_{L,i} \text{ if } \lambda_i \ge 0 \\
        \psi_{R,i} \text{ if } \lambda_i < 0
    \end{cases}
\end{equation}
This is essentially first-order upwind on the charecteristic velocities.
The flux on the interface is then
\begin{equation}
    F = S \Lambda \tilde{\psi}_i.
\end{equation}

I do it a little bit differently in CFDBox.  I decompose the flux itself
within each control volume along the eigenstates and then take the upwind
components to define the flux along the interface.

\end{document}

