\documentclass{doc}

\usepackage{tikz}
\usetikzlibrary{cd, matrix, calc, arrows}

\title{Derivation of the Isentropic Speed of Sound, Bernoulli's Principle, and Normal Shock Equations}
\author{P. Kannan}
\classification{}
\logoraw{}


\begin{document}

\begin{rk}
The inviscid 1D advection equation describes the motion of an
abstract wave.  It moves the initial solution ``to the right'' by
a charecteristic velocity.
\end{rk}

\begin{thm}[Solution of the 1D Inviscid Advection Equation]
Let $\phi:\R^2\to \R$ map $(t,x)\mapsto \phi(t,x)$.
Let the charecteristic velocity $\alpha \in \R$.
Given initial conditions $\phi(0,x)$, the solution to the advection
equation
\begin{equation}
    \pd{\phi}{t} + \alpha \pd{\phi}{x} = 0.
\end{equation}
is
\begin{equation}
    \phi(t,x) = \phi(0,x-\alpha t).
\end{equation}
\end{thm}
\begin{proof}
The advection equation is solved by the ``method of charecteristics".
Suppose the solution $z=\phi(t,x)$ was known.  The solution surface can
be described by
\begin{equation}
    f(t,x,z) = \phi(t,x) - z = 0.
\end{equation}
The original PDE can now be written as
\begin{equation}
0 = \pd{f}{t} \dd{t}{\tau} + \pd{f}{x} \dd{x}{\tau} + \pd{z}{x} \dd{z}{\tau}
    = \begin{bmatrix}
        \pd{f}{t} & \pd{f}{x} & \pd{f}{z}
    \end{bmatrix}
    \begin{bmatrix}
        1 \\
        \alpha \\
        0
    \end{bmatrix}.
\end{equation}

Let $t(\tau)$, $x(\tau)$ and $z(\tau)$ parameterize any curve on the solution 
surface.  Then, because $f(t(\tau),x(\tau),z(\tau))=0$ for all $\tau$,
\begin{align}
    0 &= \dd{f}{\tau} = \pd{f}{t} \dd{t}{\tau} + \pd{f}{x} \dd{x}{\tau} + \pd{z}{x} \dd{z}{\tau}
    = \begin{bmatrix}
        \pd{f}{t} & \pd{f}{x} & \pd{f}{z}
    \end{bmatrix}
    \begin{bmatrix}
        \dd{t}{\tau} \\
        \dd{x}{\tau} \\
        \dd{z}{\tau}
    \end{bmatrix}.
\end{align}
Therefore 
\begin{align}
    \dd{}{\tau} \begin{bmatrix}
        t \\
        x \\
        z
    \end{bmatrix} &= 
    \begin{bmatrix}
        1 \\
        \alpha \\
        0
    \end{bmatrix}.
\end{align}
\end{proof}
Define $t(\tau=0)=0$, $x(\tau=0)=x_0$, $z(\tau=0)=\phi(0,x_0)$.
Solving the ODE with the initial conditions gives
\begin{align}
    t &= \tau. \\
    x &= \alpha \tau + x_0= \alpha t + x_0. \\
    z &= \phi(0,x_0).
\end{align}
Then,
\begin{equation}
    \phi(0, x_0) = z(\tau) = \phi(t(\tau),x(\tau)) = \phi(t,\alpha t + x_0).
\end{equation}
With some more abuse of notation, define $x=\alpha t+x_0$ and this gives
\begin{equation}
    \phi(0, x-\alpha t) = \phi(t,x).
\end{equation}

\begin{rk}[Solution of Systems of Advection Equations]
    Suppose $\phi:\R^2 \to \R^n$ map $(t,x)\mapsto \phi(t,x)$.
    Let $A\in M_n(\R)$ be diagonalizable with eigendecomposition
    $A=S \Lambda S^{-1}$.  Let the system of PDEs to be solved be
    \begin{equation}
        \pd{\phi}{t} + A \pd{\phi}{x} = 0.
    \end{equation}
    By using the eigendecomposition of $A$,
    \begin{align}
        0 &= \pd{\phi}{t} + A \pd{\phi}{x} \\
        0 &= \pd{\phi}{t} + S \Lambda S^{-1} \pd{\phi}{x} \\
        0 &= S^{-1} \pd{\phi}{t} + \Lambda S^{-1} \pd{\phi}{x}.
    \end{align}
    Define eigenstates $\psi = S^{-1} \phi$.  Then,
    \begin{equation}
        0 = \pd{\psi}{t} + \Lambda \pd{\psi}{x}.
    \end{equation}
    Note that this is now a system of $n$ 1D advection equations
    each with a charecteristic velocity associated with an
    eigenvalue of $A$.  After solving for $\psi$, $\phi$
    can be obtained by substituting $\phi = S\psi$.
\end{rk}

\begin{rk}[Isentropic Equations of Flow]
The Euler Equations describe the motion of an isentropic fluid.
They can be arrived at by simplifying the Navier-Stokes equations.
$\rho$, $v$, and $u$ are density, velocity, and specific internal energy.
$r=[x;y;z]$.
\begin{enumerate}
    \item Conservation of Mass.
        \begin{equation}
            \pd{\rho}{t} + (\rho \tp{v}) \ldel_r = 0
        \end{equation}
    \item Conservation of Momentum.
        \begin{equation}
            \pd{ \rho v }{t} + (\rho v \tp{v}) \ldel_r + \rdel_r p =  0
        \end{equation}
    \item Conservation of Internal Energy Density.
        \begin{equation}
            \pd{\rho u}{t} + (\rho u \tp{v}) \ldel_r + p \tp{\rdel_r} v = 0.
        \end{equation}
\end{enumerate}
\end{rk}

\begin{thm}[The Isentropic Speed of Sound]
    The speed of sound $\lambda$ for a gas under isentropic dynamics is
    \begin{equation}
        \lambda = \sqrt{\left(\pd{p}{\rho}\right)_{q}},
    \end{equation}
    where the partial is taken adiabatically.  In particular,
    \begin{equation}
        \lambda = \sqrt{\gamma \frac{\rho}{p}}.
    \end{equation}
    and for an ideal gas,
    \begin{equation}
        \lambda = \sqrt{\gamma R T}.
    \end{equation}
\end{thm}
\begin{proof}
Let the ambient density be $\rho_0$, the ambient pressure be $p_0$,
the ambient temperature $T_0$ and the ambient momentum be zero.
Let the $\rho$, $p$ and $T$ be adiabatically perturbed an arbitrarily small amount.
Let perturbed quantities be prefixed by a $\Delta$, for example $\Delta \rho$ for
the perturbed density.
Let the perturbed pressure be $\Delta p = \left(\pd{\rho}{p}\right)_q \Delta\rho$
where $\pd{\rho}{p}$ is taken adaibatically.  Because there are no viscous effects or
heat transfer (isentropic dynamics), the 1D isentropic equations of flow are
\begin{enumerate}
    \item Conservation of Mass.  Let momentum $w=\rho v$.
        \begin{equation}
            \pd{\rho}{t} + \pd{w}{x} = 0.
        \end{equation}
    \item Conservation of Momentum.
        \begin{equation}
            \pd{w}{t} + \pd{ p + w^2/\rho }{x} = 0.
        \end{equation}
\end{enumerate}
Substituting $w = \Delta w$, $p=p_0 + \pd{p}{\rho} \Delta \rho$, $\rho = \rho_0 + \Delta \rho$
gives 
\begin{equation}
0 = \pd{}{t} \begin{bmatrix}
    \Delta \rho \\
    \Delta w
\end{bmatrix} +
\pd{}{x} \begin{bmatrix}
    \Delta w \\
    \frac{(\Delta w)^2}{\rho + \Delta \rho} + p_0 + \pd{p}{\rho} \Delta \rho
\end{bmatrix}.
\end{equation}
Linearizing the expression within $\pd{}{x}(\cdot)$ about $\Delta w=0$
and $\Delta \rho=0$ gives
\begin{equation}
0 = \pd{}{t} \begin{bmatrix}
    \Delta \rho \\
    \Delta w
\end{bmatrix} +
\underbrace{\begin{bmatrix}
    0 & 1 \\
    \pd{p}{\rho} & 0
\end{bmatrix}}_{A}
\pd{}{x} \begin{bmatrix}
    \Delta \rho \\
    \Delta w
\end{bmatrix}.
\end{equation}
Because $A$ has zero trace, it has eigenvalues $\lambda$ and $-\lambda$.
The product of the eigenvalues of $A$ is the determinant of $A$.  Therefore
\begin{align}
    -\lambda^2 &= -\pd{p}{\rho}. \\
    \lambda &= \sqrt{\pd{p}{\rho}}.
\end{align}
Because perturbations are performed adiabatically,  the change
in internal energy is (let $m$=mass),
\begin{align}
    \partial U &= Q - p \partial V \\
    \partial U &= -p \partial V \\
    \frac{1}{m} \partial U &= -\frac{1}{m} p \partial V
\end{align}
Substituting
\begin{align}
    \rho &= \frac{m}{V} \\
    \partial \rho &= -\frac{m}{V^2} \partial V = -\rho \frac{\partial{V}}{V} \\
    \partial V &= -V\frac{\delta\rho}{\rho} \\
    \frac{\partial V}{m} &= -\frac{\delta\rho}{\rho^2}
\end{align}
into the equations give
\begin{align}
    \partial u &= \frac{p}{\rho^2}\partial \rho \\
    c_v \partial T &= \frac{p}{\rho^2}\partial \rho.
\end{align}
The change in enthalpy is
\begin{align}
    H &= U + pV \\
    \partial H &= \partial U + p \partial V + V \partial p \\
    \partial H &= -p \partial V + p \partial V + V \partial p \\
    \partial H &= V \partial p \\
    \frac{1}{m} \partial H &= \frac{v}{m} \partial p \\
    \partial h &= \frac{v}{m} \partial p \\
    \partial h &= \frac{\partial p}{\rho} \\
    c_p \partial T &= \frac{\partial p}{\rho}.
\end{align}
Dividing equations,
\begin{align}
    \frac{c_p}{c_v} &=  \frac{\partial p}{\rho}  \frac{\rho^2}{p\partial \rho} \\
    \frac{c_p}{c_v} &= \frac{\rho \partial p}{p \partial\rho } \\
    \gamma &= \frac{\rho \partial p}{ p \partial\rho } \\
    \gamma \frac{p}{\rho} &= \pd{p}{\rho} \\
     \pd{p}{\rho} &= \gamma R T
\end{align}
Therefore,
\begin{equation}
    \lambda = \sqrt{\left(\pd{p}{\rho}\right)_{q}} = \sqrt{\gamma R T}.
\end{equation}
\end{proof}

\begin{thm}[Some Isentropic Relations for an Ideal Gas]
\begin{equation}
    \frac{p}{p_0} = \left( \frac{\rho}{\rho_0} \right)^\gamma = \left( \frac{T}{T_0} \right)^\frac{\gamma}{\gamma-1} = \left( \frac{\lambda}{\lambda_0} \right)^\frac{2\gamma}{\gamma-1}
\end{equation}
\end{thm}
\begin{proof}
From the proof for the speed of sound,
\begin{align}
    \gamma \frac{p}{\rho} &= \pd{p}{\rho} \\
    \gamma \frac{\partial \rho}{\rho} &= \frac{\partial p}{p} \\
    C_0 + \gamma \log \rho &= \log p \\
    C_1 \rho^\gamma &= p \\
    \left( \frac{\rho}{\rho_0} \right)^\gamma &= \frac{p}{p_0} \\
    \left( \frac{\rho}{\rho_0} \right)^\gamma &= \frac{\rho R T}{\rho_0 R T_0} \\
    \left( \frac{\rho}{\rho_0} \right)^{\gamma-1} &= \frac{T}{T_0} \\
    \frac{p}{p_0} &= \left( \frac{T}{T_0} \right)^\frac{\gamma}{\gamma-1} \\
    \lambda^2 &= \gamma R T \\
    T &= \frac{\lambda^2}{\gamma R} \\
    \frac{p}{p_0} &= \left( \frac{\lambda^2 \gamma R}{ \gamma R a_0^2} \right)^\frac{\gamma}{\gamma-1} \\
    \frac{p}{p_0} &= \left( \frac{\lambda}{\lambda_0} \right)^\frac{2\gamma}{\gamma-1}
\end{align}
\end{proof}

\begin{thm}[Compressible Bernoulli's Princple]
...
\end{thm}
\begin{thm}[Normal Shock Equations]
...
\end{thm}

%\begin{thm}[Bernoulli's Princple]
%Let the stagnation enthalpy $h_0\in \R$ be a constant.
%Then, for isentropic dynamics,
%\begin{equation}
%    \frac{v^2}{2} + h = h_0.
%\end{equation}
%\end{thm}
%\begin{proof}
%Total energy density $E = \rho( u+\frac{v^2}{2} )$ is a conserved quantity.
%Let $v_0=0$.  Then $\rho_0 u_0 = \rho( u+\frac{v^2}{2} )$.
%\end{proof}

\end{document}

